\documentclass[10pt,a4paper]{report}
\usepackage[latin1]{inputenc}
\usepackage[ngerman]{babel}

\author{Brinkmann, Maximilian, 2319886, maximilian.brinkmann@haw-hamburg.de \\
	Gkaitatzis, Andreas, 2320008, Gkaitatzis.A@web.de}

\title{Design Dokument \\PR1 Praktikum WS2016}
\date{ 	\begin{tabular}{|c|c|c|}
		\hline 
		Version & Datum & Anmerkung\\ 
		\hline 
		1.0 & 01.11.2016 & Erste Abgabe\\ 
		\hline 
\end{tabular} }


\begin{document}
	\maketitle
	
	\section*{Aufgabe 1:Logische Ausdr�cke}
	\subsection*{Design in Pseudocode}
	\subsubsection*{a:}
	(i $\neq$ 0) \& (j $\neq$ k) \& (k $\neq$ 0)\\
	$\rightarrow$ Ausgabe
	
	\subsubsection*{b:}
	(i $\ge$ 0) \& (i\%17 = 0)\\
	$\rightarrow$ Ausgabe
	
	\subsubsection*{c:}
	(j\%2 = 1) \& (j $<$ 40) \& (j $>$ 20)\\
	$\rightarrow$ Ausgabe
	
	\subsubsection*{d:}
	((k\%3 = 0) \& (k\%5 = 0)) or ((k\%5 = 0) \& (k\%7 = 0)) 
	
	or((k\%5 = 0) \& (k\%11 = 0))\\
	$\rightarrow$ Ausgabe
	 
	\subsubsection*{e:}
	(b = c) = d\\
	$\rightarrow$ Ausgabe
	
	\newpage
	\section*{Aufgabe 2: Messwert-Tabelle}
	\subsection*{Desgin in Pseudocode}
	\subsubsection*{Variabeln}
	int posX\\
	int posY\\
	flout temp\\
	anzZeilen\\
	schleifenZaehler
	
	\subsubsection*{Aktivit�ten}
	- Ausgabe Kopfzeile und Trennstrich\\
	$\rightarrow$ Formatierung Stellen: 11, 12, 15\\
	$\rightarrow$ Gesamtl�nge: 40
	\\\\
	- Solange schleifenZaehler $<$ anzZeilen
	\\
	$\rightarrow$ Benutzereingabe f�r posX, posY \& temp\\
	$\rightarrow$ Ausgabe posX, posY \& temp\\
	$\longrightarrow$ posX max. 11 Stellen lang, mit Vorzeichen und rechtsb�ndig\\
	$\longrightarrow$ posY max. 12 Stellen lang, mit Vorzeichen und rechstb�ndig\\
	$\longrightarrow$ temp max. 15 Stellen lang, 3 Nachkommastellen mit Vorzeichen\\
	$\rightarrow$ schleifenZaehler plus 1
	
	\newpage
	\section*{Aufgabe 3: Quadratische Gleichung}
	\subsection*{Desgin in Pseudocode}
	\subsubsection*{Variabeln}
	
	
	
\end{document}