\documentclass[10pt,a4paper]{report}
\usepackage{textcomp}
\usepackage{amsmath}
\usepackage[utf8]{inputenc}
\usepackage[ngerman]{babel}

\author{Brinkmann, Maximilian, 2319886, maximilian.brinkmann@haw-hamburg.de \\
	Gkaitatzis, Andreas, 2320008, Gkaitatzis.A@web.de}

\title{Design Dokument \\PR1 Praktikum WS2016}
\date{ 	\begin{tabular}{|c|c|c|}
		\hline 
		Version & Datum & Anmerkung\\ 
		\hline 
		1.0 & 01.11.2016 & Erster Entwurf\\ 
		\hline 
		1.1 & 10.11.2016 & Erste Korrektur\\ 
		\hline
		\end{tabular} 
	}


\begin{document}
	\maketitle
	
	\section*{Aufgabe 1:Logische Ausdrücke}
	\subsection*{Design in Pseudocode}
	\subsubsection*{a:}
	(i $\neq$ 0) \& (j $\neq$ 0) \& (k $\neq$ 0)\\
	$\rightarrow$ Ausgabe
	
	\subsubsection*{b:}
	(i $<$ 0) \& (i\%17 = 0)\\
	$\rightarrow$ Ausgabe
	
	\subsubsection*{c:}
	(j\%2 = 1) \& (j $<$ 40) \& (j $>$ 20)\\
	$\rightarrow$ Ausgabe
	
	\subsubsection*{d:}
	((k\%5 = 0) \& ((k\%7 = 0) oder(k\%7 = 0) oder (k\%11 = 0))\\
	$\rightarrow$ Ausgabe
	 
	\subsubsection*{e:}
	(b = c) \& (c = d)\\
	$\rightarrow$ Ausgabe
	\\\\
	Jeweiligen Bedingungen für eine If-Unterscheidung.
	
	\newpage
	\section*{Aufgabe 2: Messwert-Tabelle}
	\subsection*{Desgin in Pseudocode}
	\subsubsection*{Variabeln}
	int posX1\\
	int posY1\\
	double temp1\\\\
	int posX2\\
	int posY2\\
	double temp2\\\\
	int posX3\\
	int posY3\\
	double temp3\\
	
	\subsubsection*{Aktivitäten}
	- Benutzereingabe für posX1, posY1 \& temp1\\
	- Benutzereingabe für posX2, posY2 \& temp2\\
	- Benutzereingabe für posX3, posY3 \& temp3\\
	$\rightarrow$ jeweilig mit Überprüfung der Vorraussetzungen:\\
	$\longrightarrow$ x,y im Bereich -1024 bis 1024\\
	$\longrightarrow$ Temperatur -40 bis 85\\
	\\
	- Ausgabe Kopfzeile und Trennstrich\\
	\\
	$\rightarrow$ Ausgabe posX, posY \& temp\\
	$\rightarrow$ jeweils alle drei Werte ($posX_{1,2,3}$ ;  $posY_{1,2,3}$ ; $temp_{1,2,3}$
	\\\\
	$\longrightarrow$ posX max. 10 Stellen lang, mit Vorzeichen und rechtsbündig\\
	$\longrightarrow$ posY max. 10 Stellen lang, mit Vorzeichen und rechstbündig\\
	$\longrightarrow$ temp max. 10 Stellen lang, 3 Nachkommastellen mit Vorzeichen\\
	
	\newpage
	\section*{Aufgabe 3: Quadratische Gleichung}
	\subsection*{Desgin in Pseudocode}
	\subsubsection*{Variabeln}
	double a\\
	double b\\
	double c\\
	double x1\\
	double x2
	
	\subsubsection*{Aktivitäten}
	Fallunterscheidung zwischen keiner, einer und zwei Lösungen:\\
	Keine:$b^2 - 4ac < 0$\\
	Eine: $a = 0$ \& $b\neq0$ oder   $\sqrt{b^2 - 4ac} = 0$\\
	Zwei: alle weitere Fälle\\
	\\
	Berechnung:\\
	Bei Keiner Lösung: keine Berechnung.\\
	Bei einer Lösung: $ \dfrac{-c}{b} $ bzw. $\dfrac{-b}{2a} $\\
	Bei zwei Lösungen: x1: $\dfrac{-b + \sqrt{b^2 - 4ac}}{2a} $ und x2: $ \dfrac{-b - \sqrt{b^2 - 4ac}}{2a} $.\\
	\\
	$\longrightarrow$Ausgabe nach der Berechnung.
	
	\section*{Aufgabe 4: Zahlenraten}
	\subsection*{Desgin in Pseudcode}
	\subsubsection*{Variabeln}
	int zufallsZahl\\
	int benutzerZahl\\
	
	\subsubsection*{Aktivitäten}
	- zufallsZahl mit srand und time erzeugen $\rightarrow$ \%10+1\\
	- Eingabe von benutzerZahl\\
	- Abgleichen beider Variablen, Ausgabe ob richtig, zu hoch oder zu niedrig $\rightarrow$ Angabe der Differenz\\
	\\
	Weiterführend: \\
	Nutzereingabe, sowie Abgleich mit Ausgabe, in Do-While Schleife; solange wiederholen bis zufallsZahl = nutzerZahl.
	
	
\end{document}